\documentclass[a4paper]{article}
\usepackage[pdftex]{hyperref}
\begin{document}
\title{Artigos Wikipedia \\
            \large Historia Cultura }
\maketitle
\author{Wikipedia}
\date{\today}
\setcounter{tocdepth}{1}
\tableofcontents

\newpage
\section{Como reinar Portugal}
\href{www.google.pt}{\textit{Artigo Original}}
\newline
\textbf{Categorias:}
\begin{itemize}
	\item Historia;
	\item Cultura;
\end{itemize}
\subsection{Abstract}

\begin{tabular}{|p{3cm}||p{3cm}|p{3cm}|p{3cm}|}
\hline
\multicolumn{2}{|c|}{Info} \\
\hline
\end{tabular}
ola tudo bem? 
\newpage
\section{Como reinar Portugal}
\href{www.google.pt}{\textit{Artigo Original}}
\newline
\textbf{Categorias:}
\begin{itemize}
	\item Historia;
	\item Cultura;
\end{itemize}
\subsection{Abstract}

\begin{tabular}{|p{3cm}||p{3cm}|p{3cm}|p{3cm}|}
\hline
\multicolumn{2}{|c|}{Info} \\
\hline
\end{tabular}
ola tudo bem? 
\newpage
\end{document}
